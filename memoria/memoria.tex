\documentclass[12pt]{report}
\usepackage[a4paper, total={17cm, 24cm}]{geometry}
\usepackage{fancyhdr}
\pagestyle{fancy}
\fancyhf{}
\fancyhead[L]{\leftmark}
\fancyfoot[C]{\thepage}
\setlength{\headheight}{15pt}
\renewcommand{\chaptermark}[1]{\markboth{#1}{}}

\usepackage{lipsum} % Agregar el paquete lipsum para generar texto de ejemplo

\begin{document}

\begin{titlepage}
    \begin{center}
        \vspace*{1cm}
        
        \textbf{\huge Computación Ubicua}
        
        \vspace{0.5cm}
        \textbf{\large Smart Uni}
        
        \vspace{1.5cm}
        
        \textbf{\huge Integración de sistemas IoT en la Universidad de Alcalá}
        
        \vspace{2cm}
        
        \textbf{\large Javier Lombardía Ocaña, César Martín Guijarro,}\\
        \textbf{\large Lucía Picado Joglar y Valeria Fernanda Villamares Félix}
        
        \vfill
        
        \textbf{\large Universidad de Alcalá}\\
        \textbf{\large \today}
        
    \end{center}
\end{titlepage}

\pagenumbering{arabic}

\tableofcontents


\pagenumbering{arabic}

\chapter[Introducción]{Introducción\\ \normalsize En este capítulo se va a realizar una introducción detallada sobre todos los temas}
\section{Objetivos}
\lipsum[1-2] % Generar texto de ejemplo con lipsum
\newpage
\section{Justificación}
\lipsum[3-4]
\newpage
\section{Estructura de la memoria}
\lipsum[5-6]

\chapter[Propuestas]{Propuestas de mejora con respecto al primer proyecto\\ \normalsize En este capítulo trataremos sobre todos los inconvenientes con los que nos topamos durante el desarrollo de la primera práctica y que intentaremos mejorar en el capítulo posterior.}
\section{Necesidad de estar físicamente}
\section{Alojamiento del servidor}
\section{Alojamiento de la BBDD}
\section{Eliminación de la máquina virtual} %mencionar entorno virtual
\section{Inconvenientes adicionales} %desplegar para cada cambio. Dificil comunicación con BBDD
\section{Testing lento}

\chapter{Implementación y soluciones a los problemas}
\section{Servidores locales}
\section{Implementación con FastApi}
\section{Entornos virtuales}%mejor que una máquina virtual de mierda
\section{Base de datos online} %en una página gratuita
\section{Testing con Postman} %hablar de la metodología
\section{Mejoras adicinoales} %mini librería BBDD


\chapter{Marco teórico}
\section{Computación ubicua}
\lipsum[1-2]
\newpage
\section{Internet de las cosas}
\lipsum[3-4]
\newpage
\section{Universidad de Alcalá}
\lipsum[5-6]

\chapter{Metodología}
\section{Desarrollo del proyecto}
\lipsum[1-2]
\newpage
\section{Etapas del proyecto}
\lipsum[3-4]

\chapter{Resultados}
\section{Análisis de los resultados}
\lipsum[1-2]
\newpage
\section{Discusión de los resultados}
\lipsum[3-4]

\chapter{Conclusiones}
\section{Logros alcanzados}
\lipsum[1-2]
\newpage
\section{Recomendaciones}
\lipsum[3-4]
\newpage
\section{Trabajo futuro}
\lipsum[5-6]

\end{document}
